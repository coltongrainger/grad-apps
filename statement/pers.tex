\documentclass{article}
\edef\restoreparindent{\parindent=\the\parindent\relax}
\usepackage{parskip} 
\restoreparindent
\usepackage[margin=1in]{geometry}

\usepackage[
  style=ieee,
  backend=biber,
  sorting=none
]{biblatex}
\addbibresource{pers.bib}

\usepackage{marvosym}
\usepackage{fontspec} 
\setmainfont{Georgia}
\usepackage[
  colorlinks=true,       % false: boxed links; true: colored links
  citecolor=black,       % color of links to bibliography
  urlcolor=blue          % color of external links
]{hyperref} 

\pagestyle{empty}

\newcommand\textbox[1]{\parbox{.33\textwidth}{#1}} 

\newcommand\printheader{
  \noindent 
  \textbox{\emph{Statement of Purpose}} 
  \textbox{\centerline{\large\bf Colton Grainger}} 
  \textbox{\raggedleft \texttt{coltongrainger@gmail.com}\ 
}

\vspace{-14pt} 
\noindent 
\rule{\textwidth}{1pt}\\ }

\begin{document} 

\printheader

The robustness of a mathematical method determines its utility. Just imagine designing a communication network that fails to account for topological perturbations, or modeling an epidemic with strictly deterministic differential equations. My goal is to research robust methods in algebraic topology that can be applied in both abstract and computational settings.

For example, consider training an AI to distinguish the tone signature of different musical instruments. Applying persistent homology, we associate ``holes'' in an audio recording's time-delay phase-space with a sample statistic: the persistent rank function (PRF). Corresponding informally with Nikki Sanderson \cite{Sanderson17}, I learned a computer trained on PRFs more accurately classifies tones than one trained on Fast Fourier Transforms. Here a topological invariant answers the question ``which instrument?'' with higher fidelity than a recording's frequency transform.

My research interest stems from my exposure to homotopy theory as an undergraduate, my grounding in point-set topology, and insight from two years of service work graduating.

Advised by Dr.~Jonny Comes, my senior independent study examined how Galois theory constrains the solution space of Fuchsian-type differential equations. Following Michio Kuga, we developed a correspondence between the fundamental group of \(D = \mathbf{C}\setminus\{x_1,\ldots,x_n\}\) and this surface's universal covering space \(\tilde{D}\). Exploiting the representation of the group of covering transformations \(\Gamma(\tilde{D} \to \tilde{D})\) as a group of linear automorphisms, I parameterized solutions to the hypergeometric DE. For interesting cases, I found the monodromy representation at singular points, and generated plots. The linked poster \cite{Grainger16} summarizes my method, its history, and discusses applications to mathematical physics. At the same time, I studied point-set topology under Dr.~Dave Rosoff, who led seminar in the Moore method. I reasoned from counter examples, occasionally contributing a stronger hypothesis to our notes to account for non-intuitive spaces. We had enough time to abstract spaces to objects and maps to morphisms as a means of analogously introducing category theory from topology. I am enthusiastic to build from this ground to higher results, one of which I reached in my senior study, another of which Nikki Sanderson demonstrated ahead of me.

In addition to my academic background, I have \emph{methodological} insight from two years of service work: I know how to collaborate, how to make cumulative progress, and how to maintain a transparent workflow. In Houston, under Shaoli Bhadra, I developed scalable \href{https://github.com/coltongrainger/ymca-resources}{resources} for refugee case management, which included crowd-sourcing a \href{https://drive.google.com/open?id=1kk9yn6-4nifHLIf2tGYbW_7PiYo\&usp=sharing}{map of clinics and languages spoken} through the Google Maps API. I wrote bug reports for the implementation three databases, and, when Texas cut funding for Refugee Medical Assistance, I contributed to a data management plan for the small refugee population transitioning from state to federal medical care. In Olympia, I am a community organizer at a 24/7 homeless shelter. I rely on distributed version control, and have therefore become a staunch advocate for ``deploying early and often''. For example, I built \mbox{\href{http://volunteer.fscss.org}{volunteer.fscss.org}} to maintain a schedule of events, but it now doubles as a wiki. As I have begun mentoring interns and work-study students who are preparing to embark on careers in social work, I am noticing skills I think will transfer neatly to a teaching assistantship. In all, I have cultivated working methods I believe to be of value in the scientific disciplines: collaboration allows me to focus my effort on tasks where I have a comparative advantage, cumulation allows me to work where others have left off, and transparency allows others to work off of me.

I am serious, however, to pursue a career in mathematics. To prepare for graduate study, I have enrolled in an affordable selection of correspondence courses: Probability, Differential Equations, and Numerical Analysis. I am also reading from Hatcher's \emph{Algebraic Topology}, surveying topics in Gower's \emph{Companion to Mathematics}, and building a base of computing skills in the UNIX philosophy.

While I am open to the breadth of mathematical inquiry at the University of Rochester, I would like to ``push hard on a clear signal'' and study algebraic topology---specifically homotopy theory and homology groups---with computational topology in mind for applications.  I would be glad to collaborate with Profs Pakianathan, Iosevich, or Ravenel. Having surveyed these folks' recent publications and webpages, I believe the doctoral program in mathematics at the University of Rochester would be a great fit for my interests.

\printbibliography

\end{document}
