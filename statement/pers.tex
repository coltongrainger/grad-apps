\documentclass{article}
\edef\restoreparindent{\parindent=\the\parindent\relax}
\usepackage{parskip}
\restoreparindent
\usepackage[margin=1in]{geometry}

\usepackage{marvosym}
\usepackage[colorlinks = true, urlcolor = blue]{hyperref}
\usepackage{fontspec} 
\setmainfont{Georgia}

\newcommand\textbox[1]{\parbox{.33\textwidth}{#1}} 

\newcommand\printheader{\noindent
  \textbox{Statement of Purpose}
  \textbox{\centerline{\large\bf Colton Grainger}}
  \textbox{\raggedleft \href{mailto:coltongrainger@gmail.com}{\texttt{coltongrainger@gmail.com}}\ }

  \vspace{-15pt}
  \noindent
  \rule{\textwidth}{1pt}\\
}

\begin{document} 

\printheader

The robustness of a mathematical method determines its utility. Just
imagine designing a communication network that fails to account for
topological perturbations, or modeling an epidemic with strictly
deterministic differential equations. My goal, broadly, is to study and
apply algebraic topology in abstract and computational settings. 

For example, consider training an AI to distinguish the tone signature
of different musical instruments. Applying persistent homology, we
associate holes in an audio recording's time-delay phase-space with a
sample statistic: the persistent rank function (PRF). Corresponding with Nikki
Sanderson, I learned a computer trained on PRFs more
accurately classifies tone signatures than one trained on Fast Fourier Transforms. 
Here, computational topology answers the question ``which instrument?'' with higher 
fidelity than frequency analysis.

My research interest stems from (i) my exposure to topology and its
applications as a college senior, and (ii)~insight from two years of
service work since graduating.

Advised by Dr.~Jonny Comes, my senior independent study\footnote{C.
  Grainger,
  \href{http://coltongrainger.com/documents/cgrainger_coursework_galois_poster.pdf}{Applications
  of Galois Theory: Monodromy Groups and Fuchsian DiffEqs}, College of Idaho SRC, 2016.}
examined how Galois theory constrains the solution space of
Fuchsian-type DiffEqs. Following Michio Kuga, we developed a
correspondence between the fundamental group on
\(D = \mathbf{C}\setminus\{x_1,\ldots,x_n\}\) and this region's
universal covering space \(\tilde{D}\). Exploiting the representation of
the group of covering transformations
\(\Gamma(\tilde{D} \to \tilde{D})\) as a group of linear automorphisms,
I parameterized solutions to the hypergeometric DiffEq. For
interesting cases, I found the monodromy representation at singular
points, and generated plots. I presented my method, its history, and an
application to fluid flow at The College of Idaho's 2016 student
research conference. At the same time, I studied point-set topology
under Dr.~Dave Rosoff. He led seminar in a modified Moore method.
I reasoned from topological counter examples, finding errata in our notes by
discovering non-intuitive spaces, e.g., the space \(X\) whose
open sets are those sets with countable complements.
We drew out themes in topology to introduce category theory.
I am enthusiastic to build from this ground to higher results,
one of which I reached in my senior study, another of which Nikki
Sanderson demonstrated ahead of me.

I gained insight to the necessity of \emph{applications} (esp. with
positive societal impact) by completing two years of service work.
In Houston, under Shaoli Bhadra, I developed scalable
\href{https://github.com/coltongrainger/ymca-resources}{resources} for
refugee case management. I crowd-sourced a
\href{https://drive.google.com/open?id=1kk9yn6-4nifHLIf2tGYbW_7PiYo\&usp=sharing}{map
of clinics and languages spoken} with the Google Maps API. I wrote bug
reports for the implementation of a SQL case-notes database, and, when
Texas cut funding for Refugee Medical Assistance, I contributed to a
data management plan for the small refugee population transitioning from 
state to federal medical care. In Olympia, I am coordinating the volunteer program for a
24/7 homeless shelter. I rely on a distributed workflow---with \texttt{git} for version control and
pandoc markdown for administrivia. I built
\href{http://volunteer.fscss.org}{volunteer.fscss.org} to maintain a
schedule of events and to ``reply all'' to volunteers. As we rely on
interns and work-studies, I am also collaborating with the 
Evergreen State College to write service-learning syllabi. 

I am serious, however, to pursue math as my vocation. To prepare for graduate 
study, I have enrolled in correspondence courses at the University of Idaho.
By May 2018, I will have reviewed Differential Equations,  
Probability and Numerical Analysis. I am also reading from 
Hatcher's \emph{Algebraic Topology}, surveying topics in Gower's \emph{Companion to Mathematics},
and building a base of computing skills in the UNIX philosophy. 

While I am open to a breadth of mathematical inquiry at CU Boulder, 
I am most enthusiastic to develop a rigorous foundation in algebraic topology.
My career goal is to become an applied mathematician, preferablly in machine learning 
or environmental preservation. At CU Boulder, I would be glad to collaborate with
Profs Beaudry, Farsi, or Pflaum. Having surveyed their recent publications and seminar 
webpages, I believe the doctoral program at CU Boulder would be a great fit for my
interests.

Thank you for your consideration. 
\end{document}
