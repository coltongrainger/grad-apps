\documentclass{article}
\edef\restoreparindent{\parindent=\the\parindent\relax}
\usepackage{parskip}
\restoreparindent
\usepackage[margin=1in]{geometry}

\usepackage{marvosym}
\usepackage[colorlinks = true, urlcolor = blue]{hyperref}
\usepackage{fontspec} 
\setmainfont{Georgia}

\newcommand\textbox[1]{\parbox{.33\textwidth}{#1}} 

\newcommand\printheader{\noindent
  \textbox{FINANCIAL AID}
  \textbox{\centerline{\large\bf Colton Grainger}}
  \textbox{\raggedleft \href{mailto:coltongrainger@gmail.com}{\texttt{coltongrainger@gmail.com}}\ }

  \vspace{-15pt}
  \noindent
  \rule{\textwidth}{1pt}\\
}

\begin{document} 

\printheader

The robustness of a mathematical method determines its utility. Just
imagine designing a communication network that fails to account for
dynamical perturbations, or modeling an epidemic with strictly
deterministic differential equations. My goal, then, is to research
robust topological methods for data analysis and physical applied
mathematics.

For example, consider training an AI to distinguish the tone signature
of different musical instruments. Applying persistent homology, we
associate holes in an audio recording's time-delay phase-space with a
sample statistic: the persistent rank function. Corresponding with Nikki
Sanderson, I learned a computer trained on such PRFs will more
accurately classify tone signatures than one trained on FFTs. Here, a
topological invariant answers ``which instrument?'' with higher fidelity
than frequency analysis.

My research interest stems from (i) my exposure to topology and its
applications as a college senior, and (ii)~insight from two years of
service work since graduating.

Advised by Dr.~Jonny Comes, my senior independent study\footnote{C.
  Grainger,
  \href{http://coltongrainger.com/documents/cgrainger_coursework_galois_poster.pdf}{Applications
  of Galois Theory: Monodromy Groups and Fuchsian DiffEqs}, College of Idaho SRC, 2016.}
examined how Galois theory constrains the solution space \(V\) of
Fuchsian-type DiffEqs. Following Michio Kuga, I developed a
correspondence between the fundamental group on
\(D = \mathbf{C}\setminus\{x_1,\ldots,x_n\}\) and this region's
universal covering space \(\tilde{D}\). Exploiting the representation of
the group of covering transformations
\(\Gamma(\tilde{D} \to \tilde{D})\) as a group of linear automorphisms,
I parameterized solutions to the hypergeometric DiffEq. For
interesting cases, I found the monodromy representation at singular
points, and generated plots. I presented my method, its history, and an
application to fluid flow at The College of Idaho's 2016 student
research conference. At the same time, I studied point-set topology
under Dr.~Dave Rosoff. He led seminar in a modified Moore method.
I built foundations from counter examples, often finding errata in our notes by
discovering non-intuitive spaces, e.g., the space \(X\) whose
open sets are those sets with countable complements.
We explored category theory and progressed to connectivity and compactness. I am
enthusiastic to build from this ground to higher results,
one of which I reached in my senior study, another of which Nikki
Sanderson demonstrated ahead of me.

I have gained insight to the necessity of \emph{applications} (esp. with
positive impact) by completing two years of service with displaced persons.
In Houston, under Shaoli Bhadra, I developed novel, scalable
\href{https://github.com/coltongrainger/ymca-resources}{resources} for
refugee case management. I crowd-sourced a
\href{https://drive.google.com/open?id=1kk9yn6-4nifHLIf2tGYbW_7PiYo\&usp=sharing}{map
of clinics and languages spoken} with the Google Maps API. I wrote bug
reports for the implementation of a SQL case-notes database, and, when
Texas cut funding for Refugee Medical Assistance, I contributed to a
data management plan for the transition from state to federal indigent
medical care. In Olympia, I am coordinating the volunteer program for a
24/7 shelter. I cleave to a distributed workflow. I built
\href{http://volunteer.fscss.org}{volunteer.fscss.org} to manage a
schedule of community events and to ``reply all'' to FAQs. As we rely on
internships and work-studies, I am also collaborating with the 
Evergreen State College to write service-learning syllabi. 

I am serious, however, to pursue math as my vocation. While I am
grateful for the opportunity to manage projects and cultivate
transparency, I find myself constrained in social services
as an end-user of inefficient systems.
To prepare for advanced study, I have enrolled part-time in 
the University of Idaho's Engineering Outreach
program. I will complete courses in Probability, Ordinary Differential
Equations, and Numerical Methods.  Independently, I am
learning to process data with UNIX tools and to write 
expositions in the Juptyer notebook. This computational literacy leverages 
my mathematical ability.

Though I am open to a variety of research at CU Boulder,
my experience with messy data-sets and my appetite for topology leads
naturally into topological data analysis. I would be enthusiastic to
collaborate with Prof.\ Meiss (APPM), Prof.\ Bradley (CSCI) and Prof.\ Agnes
(MATH). I see fruitful work to be done with TDA in signal processing and
network analysis; I am also curious to study higher-dimensional data in
material science. Having read through
recent publications from the above faculty, I get the sense that the
doctoral program at CU Boulder would be a great fit for my
interests.
\end{document}
