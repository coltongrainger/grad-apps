\documentclass[10pt]{letter}
\usepackage{fontspec}
\setmainfont{Georgia}
\usepackage[margin=1in,headheight=0pt,headsep=0pt]{geometry}
\addtolength{\topmargin}{-25pt}
\addtolength{\textheight}{120pt}
\date{}
\usepackage[super]{nth}
\usepackage{enumitem}

\newcommand{\topic}[1]{\textmd{#1}}
\newcommand{\term}[1]{\textmd{#1}}

\title{pers-cuboulder}

\begin{document}
\begin{letter}{Department of Applied Mathematics\\University of Colorado at Boulder}

\opening{Committee Member,}
I aim to enroll at CU Boulder to become quantitatively literate. As I want to examine how we, humans, co-create our environment, I am attracted CU Boulder's research groups in dynamical systems and stochastic processes. Upon completion of a master's degree, I plan to complete a Ph.D. and enter an ecological industry.

Here are two motivated descriptions of my research interests. 

\begin{description}[topsep=0pt]
	\item[Irrigation Regimes]
        In Idaho's Treasure Valley, farmers use a network of reservoirs and canals to suspend and divert the Boise river. To understand how flood irrigation sweeps up and transports soil, I would numerically simulate water's energy and \topic{turbidity} in flood-irrigated fields. Following up, I would examine how turbid water settles. Considering a manufactured reservoir as a dynamical system, I could determine which nutrients and fertilizers are pressured into the ground.
	\item[Ground Water Contamination]
		With the Army's decision not to grant an easement for the Dakota Access Pipeline (DAPL), I have a redoubled interest in \topic{contaminant diffusion}. If I were contributing to an environmental impact statement for DAPL, I would model (i) stresses on the pipeline as the result of geomorphism and (ii) probable hydrocarbon dispersion into ground water at points of stress. Appending stochastic analyses to this model could quantify ecological risk and economic liability.
\end{description}

I share two examples of my relevant research experience.

\begin{description}[topsep=0pt]
    \item[Galois Theory \& Fuchsian Equations] 
    	Following Michio Kuga's analysis of Fuchsian-type differential equations, I parameterized the solution space of the hypergeometric equation. For 5 interesting cases, I found the monodromy representation at singular points. I presented my method, its history and a potential application to \topic{fluid flow} at The College of Idaho's 2016 student research conference.
    \item[Igneous Dikes in Scotland]
        Relying on N.~L.~Bowen's \emph{The Evolution of the Igneous Rocks}, I modeled the cooling of plagioclase feldspar magma. I proposed that my geology abroad group in Scotland visit Glen Sligachan, a significant site for Bowen's field observations. On June \nth{4}, noticing rough shards of buoyantly exposed olivine lodged within dense clusters of plagioclase crystals, we validated Bowen's hypothesis that molten plagioclase carried partially solidified mafic minerals into the crust.
\end{description}

As I would like to be considered for a teaching assistantship, I will summarize what has prepared to teach. 

\begin{description}[topsep=0pt]
	\item[Tutoring \& Grading] I \term{tutored calculus students} one-on-one and \term{graded physics coursework}. I guided small groups through problems in elementary electromagnetism. I heard out my peers in introductory topology and posed constructive questions. As a Heritage Scholar at The College of Idaho, I led discussions in colloquium. In seminar, I organized half-hour workshops on the logistic equation and the heat equation. I also delivered an hour presentation on \term{epidemiological modeling}. 
	\item[Trying Something Different] I learned \LaTeX\ to typeset worksheets and review guides for Calculus I, Electricity \& Magnetism and Topology. I volunteered on a ranch to improve my \term{German}. I taught myself \term{Python}. I experimented with G.~Polya's guided problem solving and R.~L.~Moore's inquiry-based method. This year, clients with limited English proficiency have \emph{taught me} new ways to collaborate. 
\end{description}

Presently, I am a fellow in the Texas Episcopal Service Corps. I live in Houston with two other fellows and work as a refugee medical care intern. This work demonstrates extraordinary qualifications: I advocate for clients in emergencies and help them navigate the U.S.\ health-care system. As well, I have facilitated a transition of client data into SQL and committed to uploading our assistance resources to an online repository.

I am confident that I would contribute formidably to your program. Thank you for your consideration.

\closing{Respectfully Submitted,\\Colton Grainger}
\end{letter}
\end{document}
