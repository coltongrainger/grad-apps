\documentclass[10pt]{letter}

\usepackage{fontspec}
\setmainfont{Georgia}

\usepackage[margin=1in,headheight=0pt,headsep=0pt]{geometry}
\addtolength{\topmargin}{-52pt}
\addtolength{\textheight}{150pt}

\date{}
\usepackage[super]{nth}
\usepackage{enumitem} 

\newcommand{\topic}[1]{\textmd{#1}}
\newcommand{\term}[1]{\textmd{#1}}

\title{pers-csu}

\begin{document}
\begin{letter}{
%Department of Mathematics\\
%Colorado State University
}
\opening{Committee Member,}
I aim to enroll at Colorado State University to become quantitatively literate. J.~Liu's numerical approach to flow in porous media and P.~D.~Shipman's experience advising graduate research motivates this application. At CSU, I would design and implement numerical methods to model aquifers in the Pacific Northwest. Upon attainment of a master's degree, I plan to complete a Ph.D. and enter an ecological industry.

Here are two descriptions of my research interests. 
\begin{description}[topsep=0pt]
	\item[Sediment Transport]
		In Idaho's Treasure Valley, farmers use a network of reservoirs and canals to suspend and divert the Boise river. To understand how this irrigation regime sweeps up and transports material, I would model water's energy in flood irrigated fields. Constrained by agricultural machinery and topography, I would search for furrow patterns that minimize water's turbidity. As a related project, I would consider canal geometries that interrupt high-velocity flows.
	\item[Ground Water Contamination]
		The Army's December decision to not grant an easement for the Dakota Access Pipeline encourages me to research contaminant diffusion. In contribution to an environmental impact statement, I would (i) model geomorphic stress on the pipeline and (ii) consider the effects of a leak in regions of stress. I imagine the first item, characterizing tension in surrounding media, to be accessible as an inverse problem. I would approach the second item, assessing diffusion from an uncertain source, with a modified finite element method.
\end{description}

I share two examples of my relevant research experience. 
\begin{description}[topsep=0pt]
    \item[Galois Theory \& Fuchsian Equations] 
    	Following Michio Kuga's analysis of Fuchsian-type differential equations, I parameterized the solution space of the hypergeometric equation. For interesting cases, I found the monodromy representation at singular points. I presented my method, its history and a potential application to \topic{fluid flow} at The College of Idaho's 2016 student research conference.
    \item[Igneous Dikes in Scotland]
        Relying on N.~L.~Bowen's \emph{The Evolution of the Igneous Rocks}, I modeled the cooling of plagioclase feldspar magma. I proposed that my geology abroad group in Scotland visit Glen Sligachan, a significant site for Bowen's field observations. On June \nth{4}, noticing rough shards of buoyantly exposed olivine lodged within dense clusters of plagioclase crystals, we validated Bowen's hypothesis that molten plagioclase carried partially solidified mafic minerals into the crust. 
\end{description}

Consider my candidacy for a teaching assistantship. I here summarize what has prepared me to teach. 
\begin{description}[topsep=0pt]
	\item[Tutoring \& Grading] I \term{tutored calculus students} one-on-one and \term{graded physics coursework}. I guided small groups through problems in elementary electromagnetism. I heard out my peers in introductory topology and posed constructive questions. As a Heritage Scholar at The College of Idaho, I led discussions in colloquium. In seminar, I organized half-hour workshops on the logistic equation and the heat equation. I also delivered an hour presentation on \term{epidemiological modeling}. 
	\item[Time Away from School] Over the last year, I volunteered on a ranch outside of Stuttgart and worked at a refugee resettlement office in Houston. These experiences refined my teaching ability. For example, while I learned \LaTeX\ for mathematical exposition, with it, I have created bus guides and applications for indigent health-care. As a second example, while I was exposed to guided problem solving (G.~Polya) and inquiry based learning (R.~L.~Moore) in college, I have applied these pedagogies across language barriers. I plan ahead, relax in person, and invite questions.
\end{description}

Presently, I am a fellow in the Texas Episcopal Service Corps and employed as a refugee medical care intern. I work on a small team to provide intensive case management for refugees with complex medical conditions during their first year in the United States. In this work, I help limited English proficiency clients navigate one of the nation's densest health-care bureaucracies, I coordinate health plans to ensure coverage of medical services, and I accompany clients to safety nets (e.g., shelters and food pantries) in emergency situations.

I am confident that I would contribute formidably to your program. Thank you for your consideration.

\closing{Respectfully,\\Colton Grainger}
\end{letter}
\end{document}