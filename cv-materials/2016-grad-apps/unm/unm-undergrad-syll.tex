\documentclass[10pt]{article}
\usepackage[margin=1.25in]{geometry}
\usepackage{enumitem}
\usepackage{bibentry}
\nobibliography*
\usepackage{fontawesome}

\usepackage{fontspec} 
\setmainfont{Georgia}

\newcommand\textbox[1]{\parbox{.333\textwidth}{#1} } 
\newcommand\job[3]{\textbf{#1}\hfill {#2}\\ \emph{#3} } 
\newcommand\degree[3]{\textbf{#1}\hfill {#2}\\ \emph{#3} } 
\newcommand\cs[1]{\textsl{#1} } \newcommand\ri[1]{\textsl{#1} }
\newenvironment{details}{\begin{itemize}[itemsep=0.6pt,topsep=2.2pt] }{\end{itemize} } 

\newcommand{\entry}[7]{
	\subsubsection*{#1 #2}
	#4 credit(s). Taught by #5, with final grade \textbf{#3}.
	\begin{itemize}
    	\item #6
    	\item #7
    \end{itemize}
}

\newcommand{\indentry}[7]{
	\subsubsection*{#1 #2}
	#4 credit(s). Advised by #5, with final grade \textbf{#3}.
	\begin{itemize}
    	\item #6
    	\item #7
    \end{itemize}
}

\newcommand{\spentry}[7]{
	\subsubsection*{#1 #2}
	#4 credit(s). Taught by #5.
	\begin{itemize}
    	\item #6
    	\item #7
    \end{itemize}
}

\begin{document}
%\maketitle

\noindent\textbox{\emph{Assistantship Interest }\hfill}\textbox{\hfill{\large\textbf{Colton Grainger}\hfill}}

\vspace{-9pt}
\noindent\rule{\textwidth}{1pt}

\vspace{-9pt}
\begin{flushright} 
(208)~585-7373\\ coltoncgrainger@gmail.com\\ \faTwitter\ \faGithub\hspace{1ex} {\tt @ColtonGrainger} 
\end{flushright}
\vspace{-10pt}

\section*{Summary of Teaching Experience}
Consider my candidacy for a teaching assistantship. I summarize what has prepared me to teach. 
\begin{description}
	\item[Tutoring \& Grading] I tutored calculus students one-on-one and graded physics coursework. I guided small groups through problems in elementary electromagnetism. I heard out my peers in introductory topology and posed constructive questions. As a Heritage Scholar at The College of Idaho, I led discussions in colloquium. In seminar, I organized half-hour workshops on the logistic equation and the heat equation. I also delivered an hour presentation on epidemiological modeling. 
	\item[Time Away from School] Over the last year, I volunteered on a ranch outside of Stuttgart and worked at a refugee resettlement office in Houston. These experiences refined my teaching ability. For example, while I learned \LaTeX\ for mathematical exposition, with it, I have created bus guides and applications for indigent health-care. As a second example, while I was exposed to guided problem solving (G.~Polya) and inquiry based learning (R.~L.~Moore) in college, I have applied these pedagogies across language barriers. I plan ahead, relax, and invite questions.
\end{description}

\section*{Undergraduate Syllabi}
While I am a quick study, I want to front up my established strengths. Here I list my undergraduate course-work, declare my grade, cite the textbook and abstract the course's content. My strength in analysis and topology suggests that I would be a excellent assistant for calculus sequence courses. However, perhaps working as an assistant for introductory linear algebra or ordinary differential equations would be a fruitful learning experience for myself and others.
\subsection*{2015--2016}

\entry{MAT-441}{Topology}{A}{3}{Dave Rosoff}{\bibentry{rosoff2016}}{An introduction to the techniques and theorems of point-set topology. Approached in a modified Moore method, with emphasis on writing, revising and presenting proofs. Topics included cardinality, separation axioms, compactness, connectedness, continuity, as well as novel proofs for the Heine-Borel theorem and the fundamental theorem of algebra.}

\indentry{MAT-494}{Galois Theory for Diff Eqs}{A}{2}{Jonny Comes}{\bibentry{kuga1993}}{An independent study. Explored the correspondence between the fundamental group of the plane with $n$ points removed and its covering surface. Used Galois theory to prune the ring of continuous functions (defined out of the covering surface) down to exactly those functions that were solutions to Fuchsian type differential equations.}

\entry{PHY-330}{Electricity \& Magnetism}{A}{3}{James Dull}{\bibentry{griffiths1999}}{A survey of classical electro-magnetic theory including electrostatic and magnetostatic fields and potentials, Gauss's flux theorem, Laplace's equation, dielectrics, vector potentials, magnetization and Maxwell's equations. Focused on spoken delivery. Concluded in an oral exam.}

\entry{MAT-370}{Geometry}{A}{3}{Jonny Comes}{\bibentry{hitchman2009}}{A preparation for Felix Klein's \emph{Erlangen} program. Developed geometry in terms of a space and a group of transformations of that space. Emphasis on congruence relations. Unpacked the theory of complex functions in relation to hyperbolic geometry.}

\entry{MAT-451}{Real Analysis}{A-}{3}{Jonny Comes}{\bibentry{abbott2015}}{Proceeded from the Axiom of Completeness to rigorously prove results about the convergence of sequences and series. Defined continuity (Lipschitz and uniform), the derivative and nowhere differentiable functions. Used suprema and infima to define the Riemann integral.} 

\entry{MAT-498}{Upper Division Seminar}{A}{1}{Dave Rosoff}{\bibentry{shier1999}}{A student-led seminar concerned with computational methods for mathematical modeling. Emphasized the importance of audience understanding. With the early outbreak of HIV in Houston as a case study, I presented a introduction to epidemiological modeling.}

\entry{PHY-313}{Thermal Physics}{A}{3}{James Dull}{\bibentry{schroeder2000}}{Physical basis and applications of thermodynamics and statistical mechanics including temperature, heat heat engines, entropy and free energy. Included an introduction to Maxwell-Boltzmann, Bose-Einstein, and Fermi-Direct statistics and their application to the solution of thermal, mechanical and electrical problems in fluids and solids.}

\subsection*{2014--2015}

\spentry{MAT-431}{Complex Variables}{WA}{0}{Dave Rosoff}{No meaningful text. %\bibentry{beck2014}
}{Due to a health concern in late spring, I \textbf{administratively withdrew} from all courses.}

\spentry{PHY-400}{Quantum Physics}{WA}{0}{Kathrine Devine}{No meaningful text. %\bibentry{griffiths2005}
}{Due to a health concern in late spring, I \textbf{administratively withdrew} from all courses.}

\entry{MAT-372}{History of Mathematics}{B}{3}{Dave Rosoff}{\bibentry{boyer2011}}{A historical survey of the ideas, tools, and symbols of mathematics and the people who developed them. Contextualized sexigesimal computations, Diophantine equations and medieval number theory. Emphasis on notation and legible proofs.}

\entry{MAT-461}{Algebraic Structures}{B+}{3}{Robin Cruz}{\bibentry{clark2007}}{An inquiry based course in abstract algebra focused primarily on groups. Addressed basic properties of groups, cyclic groups, LaGrange's Theorem, homomorphisms, isomorphisms, representation theorems, normal subgroups and quotient groups. Rich with examples.}

\entry{MAT-498}{Upper Division Seminar}{A}{1}{Dave Rosoff}{\bibentry{farlow1993}}{A student-led seminar addressing partial differential equations in mathematical modeling. A prep for the COMAP contest. I presented the Fourier series solution to the heat equation.}

\entry{PHY-301}{Theoretical Mechanics}{B+}{3}{Kathrine Devine}{\bibentry{taylor2005}}{A survey of classical and modern topics in dynamics. Topics included orbital mechanics, non-inertial reference frames, rigid-body motion, Lagrangian and Hamiltonian methods, and elements of nonlinear mechanics and chaos. An introduction to Mathematica.}

\subsection*{2013--2014}

\entry{MAT-352}{Differential Equations}{C+}{3}{Dave Rosoff}{\bibentry{2008elementary}}{A study of the solution and applications of ordinary differential equations including systems of equations using matrix algebra. An introduction to SageMath.}

\entry{MAT-361}{Linear Algebra}{C}{3}{Robin Cruz}{\bibentry{breezer2012}}{A study of general vector spaces, linear transformations, eigenvalues and eigenvectors.}

\bibliographystyle{unsrt}
\bibliography{undergrad}

\end{document}

