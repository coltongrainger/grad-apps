\documentclass[10pt]{letter}

\usepackage{fontspec}
\setmainfont{Georgia}

\usepackage[margin=1in,headheight=0pt,headsep=0pt]{geometry}
\addtolength{\topmargin}{-30pt}
\addtolength{\textheight}{80pt}

\date{February 15, 2017}
\usepackage[super]{nth}
\usepackage{enumitem} 

\newcommand{\topic}[1]{\textmd{#1}}
\newcommand{\term}[1]{\textmd{#1}}

\title{pers-unm}

\begin{document}
\begin{letter}{
Department of Mathematics and Statistics\\
University of New Mexico
}
\opening{Committee Member,}
I aim to enroll at University of New Mexico to become quantitatively literate. J.~Lorenz's work in fluid mechanics and A.~Korotkevich's simulation of dynamic gas flow motivates this application. At UNM, I would design and implement numerical methods to model ground water and aquifers. I intend to defend these methods in a master's thesis. Upon attainment of an M.S., I plan to complete a Ph.D. and enter industry.

Here are two rough descriptions of my research interest. 
\begin{description}[topsep=0pt]
	\item[Sediment Transport]
		In Idaho's Treasure Valley, farmers use a network of reservoirs and canals to suspend and divert the Boise river. To understand how this irrigation regime sweeps up and transports material, I would model water's energy in flood irrigated fields. Constrained by agricultural machinery and topography, I would search for furrow patterns that minimize water's turbidity. As a related project, I would consider canal geometries that interrupt high-velocity flows.
	\item[Ground Water Contamination]
		The Army's vacillation over the Dakota Access Pipeline pushes me to research contaminant diffusion. Were I to contribute to an environmental impact statement, I would (i) model geomorphic stress on the pipeline and (ii) consider the effects of a leak in regions of stress. I imagine the first item, characterizing tension in surrounding media, to be accessible as an inverse problem. I would approach the second, modeling diffusion, with a modified finite element method.
\end{description}

I share two examples of my relevant research experience. 
\begin{description}[topsep=0pt]
    \item[Galois Theory \& Fuchsian Equations] 
    	Following Michio Kuga's analysis of Fuchsian-type differential equations, I parameterized the solution space of the hypergeometric equation. For interesting cases, I found the monodromy representation at singular points. I presented my method, its history and a potential application to \topic{fluid flow} at The College of Idaho's 2016 student research conference.
    \item[Igneous Dikes in Scotland]
        Relying on N.~L.~Bowen's \emph{The Evolution of the Igneous Rocks}, I modeled the cooling of plagioclase feldspar magma. I proposed that my geology abroad group in Scotland visit Glen Sligachan, a significant site for Bowen's field observations. On June \nth{4}, noticing rough shards of buoyantly exposed olivine lodged within dense clusters of plagioclase crystals, we validated Bowen's hypothesis that molten plagioclase carried partially solidified mafic minerals into the crust. 
\end{description}

I am now a medical care intern at a resettlement office in Houston. I work on a small team to support refugees with complex medical conditions. In this work, I help limited English proficiency clients navigate one of the nation's densest health-care bureaucracies, I coordinate health plans to ensure coverage of medical services, and I accompany clients to safety nets (e.g., shelters and food pantries) in emergency situations. 

However, my heart's work is empirical, not service-oriented. Keeping in mind that ``we can \ldots\ only augur well for the sciences when the ascent [proceeds] by a true scale and successive steps, without interruption or breach, from particulars,''\footnote{Francis Bacon. \emph{Novum Organum}. Book I, \S 104} I am ready to endure the rigors of graduate study. Eventually, I hope to have some serious ecological impact by attacking water use controversies with numerical methods. 

I would contribute formidably to your program. Thank you for your consideration.

\closing{Respectfully,\\Colton Grainger}
\end{letter}
\end{document}
