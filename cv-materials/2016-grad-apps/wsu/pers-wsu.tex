\documentclass[10pt]{letter}

\usepackage{fontspec}
\setmainfont{Georgia}

\usepackage[margin=1in,headheight=0pt,headsep=0pt]{geometry}
\addtolength{\topmargin}{-55pt}
\addtolength{\textheight}{150pt}

\date{}
\usepackage[super]{nth}
\usepackage{enumitem} 

\newcommand{\topic}[1]{\textmd{#1}}
\newcommand{\term}[1]{\textmd{#1}}

\title{pers-wsu}

\begin{document}
\begin{letter}{
%Department of Mathematics and Statistics\\
%Washington State University
}
\opening{Committee Member,}
I aim to enroll at Washington State University to become quantitatively literate. As I want to explore how we, humans, co-create our environment, I am attracted to Tom Asaki's research in applied optimization and Hong-Ming Yin's research of inverse problems. Upon completion of a master's degree, I plan to complete a Ph.D. and enter an ecological industry. Please consider my candidacy for a teaching assistantship.

Here are two motivated descriptions of my research interests. 
\begin{description}[topsep=0pt]
	\item[Sediment Transport]
		In Idaho's Treasure Valley, farmers use a network of reservoirs and canals to suspend and divert the Boise river. To understand how this irrigation regime sweeps up and transports material, I would simulate water's energy in flood irrigated fields. Constrained by available machinery and topography, I would search for furrow patterns that minimize water's turbidity. As a separate, more dynamical project, I would consider canal geometries that interrupt high-velocity flows.
	\item[Ground Water Contamination]
		With the Army's decision to not grant an easement for the Dakota Access Pipeline, I have a redoubled motivation to study contaminant diffusion. Were I contributing to an environmental impact statement, I would (i) consider geomorphic stresses on the pipeline and (ii) model hydrocarbon dispersion at points of stress. I imagine the first topic---characterizing tension in the media surrounding the pipeline---to be accessible as an inverse problem. I would approach the second topic---modeling dispersion---with derivative-free optimization, finding best and worst cases.
\end{description}

I share two examples of my relevant research experience. 
\begin{description}[topsep=0pt]
    \item[Galois Theory \& Fuchsian Equations] 
    	Following Michio Kuga's analysis of Fuchsian-type differential equations, I parameterized the solution space of the hypergeometric equation. For 5 interesting cases, I found the monodromy representation at singular points. I presented my method, its history and a potential application to \topic{fluid flow} at The College of Idaho's 2016 student research conference.
    \item[Igneous Dikes in Scotland]
        Relying on N.~L.~Bowen's \emph{The Evolution of the Igneous Rocks}, I modeled the cooling of plagioclase feldspar magma. I proposed that my geology abroad group in Scotland visit Glen Sligachan, a significant site for Bowen's field observations. On June \nth{4}, noticing rough shards of buoyantly exposed olivine lodged within dense clusters of plagioclase crystals, we validated Bowen's hypothesis that molten plagioclase carried partially solidified mafic minerals into the crust. 
\end{description}

I summarize what has prepared to teach. 
\begin{description}[topsep=0pt]
	\item[Tutoring \& Grading] I \term{tutored calculus students} one-on-one and \term{graded physics coursework}. I guided small groups through problems in elementary electromagnetism. I heard out my peers in introductory topology and posed constructive questions. As a Heritage Scholar at The College of Idaho, I led discussions in colloquium. In seminar, I organized half-hour workshops on the logistic equation and the heat equation. I also delivered an hour presentation on \term{epidemiological modeling}. 
	\item[Time Away from School] In the last year, I volunteered on a ranch in Germany and worked at a refugee resettlement office in Texas. Here are two examples of how these experiences refined my teaching ability. First, while I learned \LaTeX\ to typeset proofs in analysis and topology, I have also used it to create form letters and bus guides in Arabic. Second, while I was exposed to G.~Polya's guided problem solving and R.~L.~Moore's inquiry based method in college, I have applied their pedagogy to my work across language barriers: I plan ahead, relax (despite misunderstanding) and ask plenty of questions.
\end{description}

Presently, I am a fellow in the Texas Episcopal Service Corps. I live in Houston with two other fellows and work as a refugee medical care intern. This work demonstrates extraordinary qualifications. I advocate for clients in emergencies and help them navigate the U.S.\ health-care system. As well, I am facilitating a transition of client data into SQL and uploading our emergency assistance resources to an online repository.

I am confident that I would contribute formidably to your program. Thank you for your consideration.

\closing{Respectfully,\\Colton Grainger}
\end{letter}
\end{document}
