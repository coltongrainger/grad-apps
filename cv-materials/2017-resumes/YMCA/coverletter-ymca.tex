\documentclass[10pt]{letter}

\usepackage{fontspec}
\setmainfont{Georgia}

\usepackage[margin=1in,headheight=0pt,headsep=0pt]{geometry}
\addtolength{\topmargin}{-30pt}
\addtolength{\textheight}{80pt}

\date{February 15, 2017}
\usepackage[super]{nth}
\usepackage{enumitem} 

\title{coverletter-ymca}

\begin{document}
\begin{letter}{
Regional Refugee Coordinator\\
YMCA of Greater Houston
}
\opening{Coordinator,}

I aim to become employed as Data Analyst in order to implement and maintain the ClientTrack database. 

From my undergraduate studies to my present clerical work at the YMCA International Services, I have 2 years of expertise in data collection and technical reporting. 

I am familiar with refugee social services because I have worked 7 months as a clerk for both the  Preferred Communities program (medical care) and the Social Adjustment Services program (case management). \footnote{I took this position through the Episcopal Service Corps. I am a volunteer at the YMCA, supported by the Episcopal Diocese of Texas. I live in an intentional community with 2 other fellows. My year committment to the Service Corps ends on July \nth{15}.} 

As a clerk between two highly impacted programs, I have had to blend my original duties (quality control and data management) with client services. 

	On one hand I have been sucessful: our departments manage 228 cases, of which, on a monthly basis, I interact with about 42. In March, I wrote around 100 case notes, detailing each client's work towards medical/social self-sufficiency. My responsibilities included (i) seeking low-income health-care, (ii) scheduling medical appointments, (iii) arranging medical transportation, (iv) providing orientation for use of public transit, (v) reconcilling outstanding medical bills, and (vi) applying for rental assistance. While managing client needs, I have also created time for vetting data during the implementation of 3 SQL databases: a database for case notes and intra-agency referral, a database for refugee medical assistance determination, and a database of health insurance information. The first database, known as the Refugee Management System, is a SQL database written in Python. %I added about 125 clients to this database. 
	Given my undergraduate experience with MySQL and Python, I took responsibility for writing bugs and tasks for the improvement of this database since October. For the second and third databases, known respectively as eRED and the RMA Portal, I have had sustained correspondence with Andrew Merville from the U.S. Committee for Refugees and Immigrants to ensure that clients were effectively rolled out of the Texas Medicaid \& Healthcare Partnership database into the RMA portal. Because some client data was corrupted during the initial transfer on February \nth{1}, many clients were not informed of the change in thier Refugee Medical Assistance provider and accumulated medical bills using expired Medicaid cards. I have worked from both sides: with Andrew to verify client data with clients to reconcile medical bills with the new insurance provider. 

	As a clerk, I inherited a CSV (comma seperated value) record of 4439 clients, which I have cleaned and vetted to the best of my ability with scripts written in Visual Basic. It is this aspect of my job that is the most frustrated and in which I believe I can have the greatest impact. For example, after slogging through physical archives for about 5,000 clients (and sorting more or less of them out into fiscal year), I realized that we nominally have about 2,000 more "open" case files. But we don't served those clients. And we likely won't ever again. But the system in place for the proper "closure" of those files in RDC was so miserable and time consuming that case managers did not bother to close files at all. It would be my perogative, as the Data Analyst for the soon to be introducted client track data base, to ensure that the end user experience of the data base is as straightforward as possible. I would be committed to running statistical anaylses to determined both under-served clients and clients whose case really ought to be closed. Further, I would pre-emtively vet and clean data to be ready to produce reports on demand from the office of refugee resettlement. I am equipped with the computation skills to make bi-directional analysis possible. ClientTrack should be a resource for time-strapped case managers and an accurate reporting mechanism for grantees.

I would contribute formidably to your program. Thank you for your consideration.

\closing{Respectfully,\\Colton Grainger}
\end{letter}
\end{document}